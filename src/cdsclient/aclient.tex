\cgidef{-mail -tex -f manpage.def }
%++++++++
% Copyright:    (C) 2008-2017 UDS/CNRS
% License:       GNU General Public License
%.IDENTIFICATION aclient.tex
%.PURPOSE        Definition of macros related to Man Pages
%.AUTHOR         Francois Ochsenbein
%.VERSION   1.0  26-Aug-1992
%----------
%=============================================================
\begin{manpage}{1}{aclient} {A standard Client} {05-March-2010}
	{client}
\def\pgm{{{\bf aclient}\ }}
\def\aserver{{{\bf aserver}}}
\syntax { \pgm 
	[{\bf --\%}]
	[{\bf --b} {\em blocksize}]
	[{\bf --bsk} {\em socket\_size}]
	[{\bf --p} {\em prompt}]
	{\em host} {\em service}
	[{\bf --U}{\em name}]
	[{\bf --P}{\em password}]
	[{\em command}$\cdots$]
  }

\mansec{Description}
\pgm is a client application of the {\bf sk}(3) functions.
It transmits {\em commands} to \aserver(1) running the specified
{\em service} on the specified {\em host}, and receives the result.

\mansec{Options}
\option {--\% } asks to encode the shell-specific characters
	{\bf < > \& ? * [ ] ( ) ;} {\em(see also the {\bf Escaping 
	conventions} below)}
\option {--b {\em blocksize}} defines the size of logical blocks 
	read from the standard input and transferred to the server. 
	The default is {\bf 4k}.
\option {--bsk {\em socket\_size}} defines the physical size of 
	socket blocks used in data transmission.
	The default can be specified via the {\tt SK\_bsk} environment
	variable; if {\tt SK\_bsk} is not defined, the default
	depends on the socket standards, generally 4K.
\option {--p {\em prompt}} defines the prompt to be displayed in
	case of interactive mode. The default is \\
	{\tt {\em host}/{\em service}}.
\option {{\em host}} designates the name of the host where the
	specified {\em service} is available.
\option {{\em service}} designates the service, either as a name appearing
	in the {\tt /etc/services}(5) file, or as a number in the
	1024--2047 range. The same {\em service} must be used by \aserver.
\option{--U{\em name}} specifies a name that can be recognized
	by the Server (see {\bf sk}(3), {\em authorisation file}).
\option{--P{\em password}} specifies the password associated to
	the {\bf --U} name (see {\bf sk}(3), {\em authorisation file}).
\option{{\em command}} specifies what has to be executed by
	{\em service} at {\em host}; a semi-colon ({\bf;}) may be
	used as a separate argument to delimit commands.
	When no {\em command} is given
	as arguments, \pgm waits for commands in the {\em standard input}.

\mansec{ Sending or Capturing the Data }
	Data are normally transferred via the {\em standard} input and
	output. A first solution therefore consists 
	in specifying the command for data transfer as {\em options}, e.g.

	\pgm {\tt {\em host} {\em service\_file} Write /tmp/copy < myinput}

	There are redirection possibilities but unlike {\bf sh}(1)
	these redirections must be specified {\em before} the command.
	The example above could be executed as:
	
	\begin{raggedright}
	\pgm {\tt {\em host} {\em service\_file} }\\
	{\tt{\em host}/{\em service\_file}$>$} {\tt <myinput Write /tmp/copy}\\
	{\tt{\em host}/{\em service\_file}$>$} \dots
	\end{raggedright}

	Note that file names may be replaced by pipes for names starting
	with the $|$ character; if blanks are embedded, the command have
	to be quoted. For instance, to write the list of files
	to a foreign file:
	\begin{raggedright}
	\pgm {\tt {\em host} {\em service\_file} }\\
	{\tt{\em host}/{\em service\_file}$>$}{\tt <"|ls -l" Write /tmp/copy}\\
	{\tt{\em host}/{\em service\_file}$>$} \dots
	\end{raggedright}

\mansec{Escaping conventions}
	Since {\bf aclient} connects to {\bf aserver} where a shell
	is executed, a special convention (similar to http queries)
	is available to {\em escape} special characters and
	define arguments with special characters to the server
	program. This convention uses {\bf\%\{...\}} to define
	parameters with special characters (like blank, asterisks, etc).


\mansec{Environment Variables}
	{\tt SK\_bsk} is used by \pgm as the default {\bf --bsk} option.

\mansec{Examples}
\begin{enumerate}
\item	Query a catalog on {\em cocat1}:\\
	{\tt\bf\fg{red4}aclient cocat1 1660 gsc1.2  -c 123.12-78.12 -r 1.5 -sr}
\item	Copy a file to cocat1\\
	{\tt\bf\fg{red4}aclient cocat1 1660 save < /etc/passwd} \\
	returns the name of the remote file
\item	List remote files having a name wich contains an asterisk: \\
	{\tt\bf\fg{red4} aclient cocat1 1660 ls *\%\{*\}*}
\item	Find the star {\bf\fg{blue3} ** STT 82AA'} in the last
	version of edited simbad: \\
	{\tt\bf\fg{red4} aclient -\% newviz 1660 sim.arg 0 "** STT 82AA'"} \\
	which gives the same result as \\
	{\tt\bf\fg{red4} aclient newviz 1660 sim.arg 0 "\%{** STT 82AA'}"}
\item	Rewind the tape {\tt/dev/nrst0} and get its status on the
	{\tt foreign} host using the {\tt rtape}(4) service.
\begin{verbatim}
aclient foreign rtape setenv TAPE /dev/nrst0 \; mt stat \; mt rew \; mt stat
\end{verbatim}
\end{enumerate}
	
\seealso{ aserver(1) pipe(2) sk(3) services(5) } 
\end{manpage}
